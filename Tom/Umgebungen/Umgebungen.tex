\declaretheoremstyle[
    			headfont=\bfseries\sffamily\color{black!70!black}, bodyfont=\normalfont,
    			mdframed={
       				linewidth=1pt,
        			rightline=false, topline=false, bottomline=false,
    			}
			]{Begruendungsbox}
\declaretheoremstyle[
    			headfont=\bfseries\sffamily\color{black!70!black}, bodyfont=\normalfont,
    			mdframed={
       				linewidth=1pt,
        			rightline=false, topline=false, bottomline=false,
    			}
			]{Erinnerungsbox}
\declaretheoremstyle[
    			headfont=\bfseries\sffamily\color{black!70!black}, bodyfont=\normalfont, numbered=no,
    			mdframed={
       				linewidth=false,
        			rightline=false, topline=false, bottomline=false,
    			}
			]{Zusammenfassungsbox}
\declaretheoremstyle[
    			headfont=\bfseries\sffamily\color{black!70!black}, bodyfont=\normalfont, numbered=no,
    			mdframed={
       				linewidth=0.5pt,
        			rightline=0.5pt, topline=false, bottomline=false,
    			}
			]{Voraussetzungsbox}
\declaretheorem[style=Voraussetzungsbox, name=Voraussetzung]{vrr}
\declaretheorem[style=Zusammenfassungsbox, name=Zusammenfassen]{zsmfs}
\newenvironment{Voraussetzung}{\begin{vrr}}{\end{vrr}}
\newenvironment{Zusammenfassen}{\begin{zsmfs}}{\end{zsmfs}}
\declaretheorem[style=Erinnerungsbox, name=Vermutung]{verm}
\declaretheorem[style=Begruendungsbox, name=Begründung]{begr}
\newenvironment{begruendung}{\begin{begr}}{\end{begr}}

\declaretheorem[style=Erinnerungsbox, name=Erinnerung]{Erinnerungsdef}
\newenvironment{Erinnerung}{\begin{Erinnerungsdef}}{\end{Erinnerungsdef}}